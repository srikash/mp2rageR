\PassOptionsToPackage{unicode=true}{hyperref} % options for packages loaded elsewhere
\PassOptionsToPackage{hyphens}{url}
%
\documentclass[]{article}
\usepackage{lmodern}
\usepackage{amssymb,amsmath}
\usepackage{ifxetex,ifluatex}
\usepackage{fixltx2e} % provides \textsubscript
\ifnum 0\ifxetex 1\fi\ifluatex 1\fi=0 % if pdftex
  \usepackage[T1]{fontenc}
  \usepackage[utf8]{inputenc}
  \usepackage{textcomp} % provides euro and other symbols
\else % if luatex or xelatex
  \usepackage{unicode-math}
  \defaultfontfeatures{Ligatures=TeX,Scale=MatchLowercase}
\fi
% use upquote if available, for straight quotes in verbatim environments
\IfFileExists{upquote.sty}{\usepackage{upquote}}{}
% use microtype if available
\IfFileExists{microtype.sty}{%
\usepackage[]{microtype}
\UseMicrotypeSet[protrusion]{basicmath} % disable protrusion for tt fonts
}{}
\IfFileExists{parskip.sty}{%
\usepackage{parskip}
}{% else
\setlength{\parindent}{0pt}
\setlength{\parskip}{6pt plus 2pt minus 1pt}
}
\usepackage{hyperref}
\hypersetup{
            pdftitle={Estimate T1 map},
            pdfauthor={Sriranga Kashyap},
            pdfborder={0 0 0},
            breaklinks=true}
\urlstyle{same}  % don't use monospace font for urls
\usepackage[margin=1in]{geometry}
\usepackage{color}
\usepackage{fancyvrb}
\newcommand{\VerbBar}{|}
\newcommand{\VERB}{\Verb[commandchars=\\\{\}]}
\DefineVerbatimEnvironment{Highlighting}{Verbatim}{commandchars=\\\{\}}
% Add ',fontsize=\small' for more characters per line
\usepackage{framed}
\definecolor{shadecolor}{RGB}{248,248,248}
\newenvironment{Shaded}{\begin{snugshade}}{\end{snugshade}}
\newcommand{\AlertTok}[1]{\textcolor[rgb]{0.94,0.16,0.16}{#1}}
\newcommand{\AnnotationTok}[1]{\textcolor[rgb]{0.56,0.35,0.01}{\textbf{\textit{#1}}}}
\newcommand{\AttributeTok}[1]{\textcolor[rgb]{0.77,0.63,0.00}{#1}}
\newcommand{\BaseNTok}[1]{\textcolor[rgb]{0.00,0.00,0.81}{#1}}
\newcommand{\BuiltInTok}[1]{#1}
\newcommand{\CharTok}[1]{\textcolor[rgb]{0.31,0.60,0.02}{#1}}
\newcommand{\CommentTok}[1]{\textcolor[rgb]{0.56,0.35,0.01}{\textit{#1}}}
\newcommand{\CommentVarTok}[1]{\textcolor[rgb]{0.56,0.35,0.01}{\textbf{\textit{#1}}}}
\newcommand{\ConstantTok}[1]{\textcolor[rgb]{0.00,0.00,0.00}{#1}}
\newcommand{\ControlFlowTok}[1]{\textcolor[rgb]{0.13,0.29,0.53}{\textbf{#1}}}
\newcommand{\DataTypeTok}[1]{\textcolor[rgb]{0.13,0.29,0.53}{#1}}
\newcommand{\DecValTok}[1]{\textcolor[rgb]{0.00,0.00,0.81}{#1}}
\newcommand{\DocumentationTok}[1]{\textcolor[rgb]{0.56,0.35,0.01}{\textbf{\textit{#1}}}}
\newcommand{\ErrorTok}[1]{\textcolor[rgb]{0.64,0.00,0.00}{\textbf{#1}}}
\newcommand{\ExtensionTok}[1]{#1}
\newcommand{\FloatTok}[1]{\textcolor[rgb]{0.00,0.00,0.81}{#1}}
\newcommand{\FunctionTok}[1]{\textcolor[rgb]{0.00,0.00,0.00}{#1}}
\newcommand{\ImportTok}[1]{#1}
\newcommand{\InformationTok}[1]{\textcolor[rgb]{0.56,0.35,0.01}{\textbf{\textit{#1}}}}
\newcommand{\KeywordTok}[1]{\textcolor[rgb]{0.13,0.29,0.53}{\textbf{#1}}}
\newcommand{\NormalTok}[1]{#1}
\newcommand{\OperatorTok}[1]{\textcolor[rgb]{0.81,0.36,0.00}{\textbf{#1}}}
\newcommand{\OtherTok}[1]{\textcolor[rgb]{0.56,0.35,0.01}{#1}}
\newcommand{\PreprocessorTok}[1]{\textcolor[rgb]{0.56,0.35,0.01}{\textit{#1}}}
\newcommand{\RegionMarkerTok}[1]{#1}
\newcommand{\SpecialCharTok}[1]{\textcolor[rgb]{0.00,0.00,0.00}{#1}}
\newcommand{\SpecialStringTok}[1]{\textcolor[rgb]{0.31,0.60,0.02}{#1}}
\newcommand{\StringTok}[1]{\textcolor[rgb]{0.31,0.60,0.02}{#1}}
\newcommand{\VariableTok}[1]{\textcolor[rgb]{0.00,0.00,0.00}{#1}}
\newcommand{\VerbatimStringTok}[1]{\textcolor[rgb]{0.31,0.60,0.02}{#1}}
\newcommand{\WarningTok}[1]{\textcolor[rgb]{0.56,0.35,0.01}{\textbf{\textit{#1}}}}
\usepackage{graphicx,grffile}
\makeatletter
\def\maxwidth{\ifdim\Gin@nat@width>\linewidth\linewidth\else\Gin@nat@width\fi}
\def\maxheight{\ifdim\Gin@nat@height>\textheight\textheight\else\Gin@nat@height\fi}
\makeatother
% Scale images if necessary, so that they will not overflow the page
% margins by default, and it is still possible to overwrite the defaults
% using explicit options in \includegraphics[width, height, ...]{}
\setkeys{Gin}{width=\maxwidth,height=\maxheight,keepaspectratio}
\setlength{\emergencystretch}{3em}  % prevent overfull lines
\providecommand{\tightlist}{%
  \setlength{\itemsep}{0pt}\setlength{\parskip}{0pt}}
\setcounter{secnumdepth}{0}
% Redefines (sub)paragraphs to behave more like sections
\ifx\paragraph\undefined\else
\let\oldparagraph\paragraph
\renewcommand{\paragraph}[1]{\oldparagraph{#1}\mbox{}}
\fi
\ifx\subparagraph\undefined\else
\let\oldsubparagraph\subparagraph
\renewcommand{\subparagraph}[1]{\oldsubparagraph{#1}\mbox{}}
\fi

% set default figure placement to htbp
\makeatletter
\def\fps@figure{htbp}
\makeatother


\title{Estimate T1 map}
\author{Sriranga Kashyap}
\date{2020-04-29}

\begin{document}
\maketitle

\hypertarget{set-working-directory}{%
\subsubsection{Set working directory}\label{set-working-directory}}

\begin{Shaded}
\begin{Highlighting}[]
\NormalTok{working_dir <-}\StringTok{ "~/mp2rage_data"}
\end{Highlighting}
\end{Shaded}

\hypertarget{provide-input-nifti-filenames}{%
\subsubsection{Provide input NIfTI
filenames}\label{provide-input-nifti-filenames}}

\begin{Shaded}
\begin{Highlighting}[]
\CommentTok{# MP2RAGE UNI}
\NormalTok{in_uni <-}\StringTok{ }\KeywordTok{paste0}\NormalTok{(working_dir, }\StringTok{"/MP2RAGE_UNI.nii.gz"}\NormalTok{)}
\end{Highlighting}
\end{Shaded}

\hypertarget{provide-output-nifti-filenames}{%
\subsubsection{Provide output NIfTI
filenames}\label{provide-output-nifti-filenames}}

\begin{Shaded}
\begin{Highlighting}[]
\CommentTok{# MP2RAGE T1 map}
\NormalTok{out_t1 <-}
\StringTok{  }\KeywordTok{paste0}\NormalTok{(working_dir, }\StringTok{"/MP2RAGE_Est_T1map.nii.gz"}\NormalTok{)}
\end{Highlighting}
\end{Shaded}

\hypertarget{provide-mp2rage-sequence-parameters-from-protocol}{%
\subsubsection{Provide MP2RAGE sequence parameters from
protocol}\label{provide-mp2rage-sequence-parameters-from-protocol}}

\begin{Shaded}
\begin{Highlighting}[]
\CommentTok{# MP2RAGE parameters}
\NormalTok{slices_per_slab <-}\StringTok{ }\DecValTok{240}
\NormalTok{slice_partial_fourier <-}\StringTok{ }\DecValTok{8} \OperatorTok{/}\StringTok{ }\DecValTok{8}

\NormalTok{mp2rage_params <-}
\StringTok{  }\KeywordTok{list}\NormalTok{(}
    \DataTypeTok{mprage_tr =} \FloatTok{5.0}\NormalTok{,}
    \DataTypeTok{flash_tr =} \FloatTok{6.9e-3}\NormalTok{,}
    \DataTypeTok{inv_times_a_b =} \KeywordTok{c}\NormalTok{(}\FloatTok{900e-3}\NormalTok{, }\FloatTok{2750e-3}\NormalTok{),}
    \DataTypeTok{flip_angle_a_b_deg =} \KeywordTok{c}\NormalTok{(}\DecValTok{5}\NormalTok{, }\DecValTok{3}\NormalTok{),}
    \DataTypeTok{num_z_slices =} \OtherTok{NULL}
\NormalTok{  )}

\NormalTok{mp2rage_params}\OperatorTok{$}\NormalTok{num_z_slices <-}
\StringTok{  }\NormalTok{slices_per_slab }\OperatorTok{*}\StringTok{ }\KeywordTok{c}\NormalTok{(slice_partial_fourier }\OperatorTok{-}\StringTok{ }\FloatTok{0.5}\NormalTok{, }\FloatTok{0.5}\NormalTok{)}
\end{Highlighting}
\end{Shaded}

\hypertarget{estimate-m0-and-t1-maps}{%
\subsubsection{Estimate M0 and T1 maps}\label{estimate-m0-and-t1-maps}}

\begin{Shaded}
\begin{Highlighting}[]
\CommentTok{# Load UNI data}
\NormalTok{nii_uni <-}\StringTok{ }\KeywordTok{readnii}\NormalTok{(in_uni)}
\NormalTok{data_uni <-}\StringTok{ }\NormalTok{nii_uni}\OperatorTok{@}\NormalTok{.Data}

\CommentTok{# Estimate M0 and T1}
\NormalTok{list_of_t1_m0 <-}\StringTok{ }\KeywordTok{mp2rage_estimate_t1_m0}\NormalTok{(}\DataTypeTok{in_uni_data =}\NormalTok{ data_uni,}
                                        \DataTypeTok{in_inv2_data =} \OtherTok{NULL}\NormalTok{,}
                                        \DataTypeTok{param_list_mp2rage =}\NormalTok{ mp2rage_params)}
\end{Highlighting}
\end{Shaded}

\hypertarget{write-outputs}{%
\subsubsection{Write outputs}\label{write-outputs}}

\begin{Shaded}
\begin{Highlighting}[]
\CommentTok{# Load NIfTI structure from UNI and write out T1 map}
\NormalTok{nii_t1 <-}\StringTok{ }\NormalTok{nii_uni}
\NormalTok{nii_t1}\OperatorTok{@}\NormalTok{.Data <-}\StringTok{ }\NormalTok{list_of_t1_m0}\OperatorTok{$}\NormalTok{t1_map}
\KeywordTok{writenii}\NormalTok{(}\DataTypeTok{nim =}\NormalTok{ nii_t1, }\DataTypeTok{filename =}\NormalTok{ out_t1)}
\end{Highlighting}
\end{Shaded}

\end{document}
